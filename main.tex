\documentclass[12pt, a4paper, oneside, titlepage]{article}
\usepackage[german,ngerman]{babel}
\usepackage[utf8]{inputenc}
\usepackage{lipsum}
\usepackage{pdflscape}

\usepackage{geometry}
\geometry{
  left=3.5cm,
  right=3.5cm,
  top=2.5cm,
  bottom=2cm,
  headsep=1cm,
  includeheadfoot
}

\usepackage[all,defaultlines=4]{nowidow}
\clubpenalty=10000 
\widowpenalty=10000
\displaywidowpenalty=10000 

\setlength{\parindent}{1em} 

\usepackage{setspace}
\onehalfspacing
\newcommand{\MSonehalfspacing}{%
  \setstretch{1.44}%  default
  \ifcase \@ptsize \relax % 10pt
    \setstretch {1.448}%
  \or % 11pt
    \setstretch {1.399}%
  \or % 12pt
    \setstretch {1.433}%
  \fi
}

\setlength{\parskip}{10pt}

\usepackage{lmodern}
\setlength{\parindent}{0pt}
\usepackage{color}
\usepackage{xcolor}
\definecolor{darkRed}{HTML}{5c0910}
\definecolor{myRed}{HTML}{81111A}
\definecolor{greyRed}{HTML}{744347}
\definecolor{darkGrey}{HTML}{424242}
\definecolor{lightGrey}{HTML}{6E6E6E}
\definecolor{SpringGreen}{HTML}{c8e27a}
\definecolor{RedOrange}{HTML}{e27a68}
\definecolor{mygreen}{RGB}{28,172,0} % color values Red, Green, Blue
\definecolor{mylilas}{RGB}{170,55,241}

\usepackage{colortbl}
\usepackage{bigstrut}
\usepackage{longtable}

\usepackage{sectsty}
\partfont{\color{darkRed}}
\sectionfont{\color{myRed}} %Farbe Hauptüberschriften
\subsectionfont{\color{greyRed}} %Farbe Unterüberschriften
\subsubsectionfont{\color{lightGrey}} %Farbe Unterunterüberschriften
\paragraphfont{\color{darkGrey}} %Farbe Paragraph

\usepackage{fancyhdr}
\pagestyle{fancy}
\renewcommand{\headrulewidth}{0.4pt}	% Liniendicke Kopf
\renewcommand{\footrulewidth}{0pt}	% Liniendicke Fuß

\usepackage{graphicx}

\usepackage{parcolumns}

\usepackage{amsmath}
\allowdisplaybreaks[3]
%\DeclareMathOperator*{\argmax}{arg\,max}

\usepackage{url}
\def\UrlBreaks{\do\/\do-}

\usepackage[hidelinks]{hyperref}

\usepackage[printonlyused]{acronym}

\usepackage[center]{caption}

\usepackage{float}                        
\newfloat{anhang}{h}{fig}              
\floatname{anhang}{Anhang}

\usepackage[framemethod=TikZ]{mdframed}
\usepackage{amssymb}    % Symbole

\usepackage{ulem}

\usepackage{pdfpages}

\usepackage[T1]{fontenc}

\usepackage{tocloft}
\renewcommand{\cftsecleader}{\cftdotfill{\cftdotsep}}
\renewcommand{\cfttoctitlefont}{\normalfont \Large \bfseries \color{myRed}}
\renewcommand{\cftloftitlefont}{\normalfont \Large \bfseries \color{myRed}}
\renewcommand{\cftlottitlefont}{\normalfont \Large \bfseries \color{myRed}}
\renewcommand{\cftpartfont}{\Large \color{darkGrey}}

\usepackage{listings}

\usepackage{xparse}

\NewDocumentCommand{\codeword}{v}{%
\texttt{\textcolor{blue}{#1}}%
}

\setlength{\emergencystretch}{1em}

\newcommand{\done}{\item[\checkmark]}
\newcommand{\crossed}{\item[$\times$]}

% Hier die persönlichen Daten eingeben:

\newcommand{\titel}{Unterweisungsentwurf - zur Ausbilder-Eignungsprüfung}
\newcommand{\studiengang}{Informatik}
\newcommand{\autor}{Max Mustermann}
\newcommand{\matrikelnr}{XXXXX}
\newcommand{\kurs}{TINF19BX}
\newcommand{\abgabe}{15.11.2021}

\begin{document}
\lstset{
	numbers=left,
	numberstyle=\footnotesize\color{black},
	tabsize=2,
	breaklines=true
}

\lstset{escapeinside={(*@}{@*)}}

\setnoclub
\setcounter{secnumdepth}{3}					% Nummerierungstiefe fürs Inhaltsverzeichnis
\fontfamily{lmss}\selectfont				%schriftart für normalen Inhalt

%-------------------------------Anfangsgedöhns-------------------------------------
\pagenumbering{Roman}						% große, römische Seitenzahlen für Titelei


%---------------Titelseite-----------


\fancyhead{}								%Header
\fancyhead[L]{\textbf{Studienbereich Informatik}\newline 
Studienrichtung Angewandte Informatik}
\rhead{\includegraphics[height=1.5cm]{images/dhbw.jpg}}
\setlength\headheight{48pt}					%headerhöhe

\fancyfoot{}

\begin{center}


\vspace{1.5cm}
\topskip0pt
\vspace*{\fill}
\section*{
Thema des Unterweisungsentwurfs}

\vspace{1.5cm}

\textbf{Unterweisungsentwurf zur Ausbilder-Eignungsprüfung}
\ \\
\ \\
\ \\

\textbf{Ausbildungsberuf}\\
\textit{Fachinformatiker (Fachrichtung Anwendungsentwicklung)}

\ \\

von \\ 
\autor 
\ \\

\ \\

\begin{parcolumns}[]{2}
  \colchunk{Abgabedatum}
  \colchunk{\abgabe}
  \colplacechunks
  
 \colchunk{Matrikelnummer}
  \colchunk{\matrikelnr}
  \colplacechunks
  
  \colchunk{Kurs}
  \colchunk{\kurs}
  \colplacechunks
  

  
\end{parcolumns}



\end{center}
\vspace*{\fill}

\pagebreak


\pagestyle{plain}
\setlength\headheight{0pt}	

%-------------------------Eidesstattliche Erklärung ---------------

\section*{Eidesstattliche Erklärung}
\begin{itshape}
Gemäß § 5 (3) der \glqq Studien- und Prüfungsordnung DHBW Technik\grqq{} vom 29. September 2017. 

\ \\ 
\noindent
Ich versichere hiermit, dass ich meinen \textbf{Unterweisungsentwurf zur Ausbilder-Eignungsprüfung} mit dem Thema \glqq Thema des Unterweisungsentwurfs \grqq{} selbstständig verfasst und keine anderen als die angegebenen Quellen und Hilfsmittel benutzt habe.
Ich versichere zudem, dass die eingereichte elektronische Fassung mit der gedruckten Fassung übereinstimmt.
\ \\
\ \\
\ \\

\hrule

Ort \hspace{1.5cm} Datum \hspace{2.5cm}   Unterschrift (\autor)

\end{itshape}

\pagebreak
\tableofcontents							% Erzeugen des Inhalsverzeichnisses

\pagebreak
%\listoffigures 							% Erzeugen des Abbildungsverzeichnisses 
%\listoftables 								% Erzeugen des Tabellenverzeichnisses


\pagebreak

%---------------------------------------Hauptinhalt---------------------------------------

\pagenumbering{arabic}						% arabische Seitenzahlen für den Hauptteil
\pagestyle{fancy}
\fancyhf{}
\fancyhead[L]{\nouppercase\leftmark}
\fancyhead[R]{\thepage}
\renewcommand{\headrulewidth}{0.5pt}
\setlength\headheight{15pt}	
\pagebreak


%-------------------------------Thema der Unterweisung------------------------------------------------------------
\section{Thema der Unterweisung}  \label{kapThemaUnterweisung}
\subsection{Begründung der Themenwahl}
\lipsum[1]
\subsection{Abgrenzung des Themas}
\subsection{Richtlinienbezug}

\pagebreak
%------------------------------Rahmenbedingunen-----------------------------------------------------
\section{Rahmenbedingungen} \label{kapRahmenbedingungen}
\subsection{Beschreibung des Lernadressaten}
\subsection{Beschreibung des Betriebs}
\subsection{Ort der Unterweisung}
\subsection{Unterweisungszeipunkt und Dauer}
\subsection{Einordnung der Unterweisung in den Gesamtzusammenhang}


\pagebreak

%-----------------------Lernziele----------------------------------------------
\section{Lernziele} \label{kapLernziele}
\subsection{Richtlernziel}
\subsection{Groblernziel}
\subsection{Feinlernziele}
\subsection{Lernbereiche}
\subsubsection*{Kognitiver Bereich}
\subsubsection*{Psychomotorischer Bereich}
\subsubsection*{Affektiver Bereich}
\subsection{Lernmethoden}
\subsection{Lernzielkontrollen}
%----------------------Ablauf der Unterweisung-------------------------------------------------
\section{Ablauf der Unterweisung} \label{kapAblaufUnterweisung}
\subsection{Ablaufplan}
\subsubsection*{Vorbereitungsphase}
\subsubsection*{Erarbeitungsphase}
\subsubsection*{Kontrollphase}
\subsubsection*{Übungsphase}
\subsection{Verwendete Lehr- und Arbeitsmittel}



%----------------------Ausblick nach der Unterweisung-------------------------------------------------
\section{Ausblick} \label{kapAusblick}
\input{chapters/Ausblick.tex}
\pagebreak
%-----------------------------------Anhang----------------------------------------------
\section{Anhang}
\begin{appendix}
\pagenumbering{Roman}						% römische Seitenzahlen für Anhang
\end{appendix}

\end{document}